\documentclass[11pt]{article}

\usepackage{graphicx}
\usepackage[utf8]{inputenc}
\usepackage{lmodern}
\usepackage[T1]{fontenc}
\usepackage{multirow}
\usepackage{hyperref}


\title{Tarea I\\ \small{Análisis de Algoritmos, Grupo 3}}
\author{
Maria Andrea Rodriguez Tastets$^{1}$, Erick Elejalde Sierra$^{2}$,\\ Cristóbal Donoso Oliva$^{3}$ Matías Medina Silva$^{4}$\\\\
\small{$^{1}$Docente a cargo de la asignatura $^{2}$Ayudante de asignatura}\\
\small{$^{3-4}$Estudiantes de pregrado}\\
\small{$^{1}$andrea@udec.cl $^{2}$erick.elejalde@gmail.com $^{3}$cdonoso94@gmail.com }\\
\small{$^{4}$matiasdmedina@udec.cl}\\
\small{$^{1-2-3-4}$Dpto. de Ingeniería Civil Informática y Ciencias de la Computación}\\
\small{Universidad de Concepción, Concepción, Chile.}\\
}
\date{7 de Abril del 2016}


\begin{document}

\maketitle

\section{Pregunta 2}
Suponga que se desea mantener un elemento en memoria con probabilidad uniforme sobre todos los elementos que vayan llegando de uno en uno. Esto se quiere hacer sin saber el número de elementos que llegará por adelantado. Se le pide escribir un algoritmo que realice esto,probando que la probabilidad de que un elemento cualquier que haya pasado pueda quedar en memoria sea efectivamente uniforme. Indicar su complejidad.\\
\subsection{Fundamento}
Sea $Memory[n]$ un arreglo de tamaño $n$ el cual está completo, y $A[m]$ un arreglo de elementos (procesos) que estan en espera de ejecución. Se necesita liberar espacio dentro de $Memory$ de manera que cada elemento tenga igual probabilidad de salir, en efecto, realizaremos las combinaciones entre los $n-1$ elementos que están en memoria (el primer elemento por si solo tiene ventaja respecto del conjunto de procesos). Este proceso de selección comienza entre un par del arreglo y crece a medida que avanzamos en los elementos.

\subsection{Código}  






\end{document}
